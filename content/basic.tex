% basic.tex - 基本信息(编辑此文件更新姓名、联系方式与意向)
% 用途:存放姓名、联系方式、意向、头像等基本信息。
% 说明:主要字段放在下方的宏中。可选字段默认被注释,需时取消注释并填写。

% ====== 必填 / 主要信息(请编辑) ======
\newcommand{\ProfileName}{陈非池}
\newcommand{\ProfilePhone}{18072015070}
\newcommand{\ProfileEmail}{neverbiasu67@163.com}
% 联系方式宏(用于页眉与 resumetitle)
\newcommand{\ProfileContact}{电话:\ProfilePhone\\ 邮箱:\href{mailto:\ProfileEmail}{\ProfileEmail}}
% 求职意向(示例:虚构角色,示例用途)
% \newcommand{\ProfileTarget}{虚拟角色(示例)}  % placeholder — 已在下方定义为真实意向
% 申请研究/方向(示例)
% \newcommand{\ProfileDirection}{星之任务与救援}  % placeholder — 已在下方定义为真实方向

% ====== 可选信息(默认注释——需要时取消注释并填写) ======
% 个人网站 / 作品集(可选)
\newcommand{\ProfileWebsite}{https://neverbiasu.github.io}
% GitHub(可选)
\newcommand{\ProfileGitHub}{https://github.com/neverbiasu}
% 现居地(可选)
% 籍贯(可选)
% \newcommand{\ProfileHometown}{浙江杭州}
% 出生年月(可选)
% 个人求职目标与方向
\newcommand{\ProfileTarget}{博士生}
\newcommand{\ProfileDirection}{计算机视觉、图像编辑、多模态生成}

% 如需显示头像,请在此处设置路径(相对工作区),例如:
% 在本文件中取消下面一行的注释并设置路径(例如:assets/avatar.jpg):
% \newcommand{\ProfilePhoto}{assets/avatar.jpg}
% 如果你的 avatar.png 文件损坏或不是有效的 PNG/JPG,会导致 xelatex 构建失败。
% 将下面一行取消注释并替换为有效图片路径可显示头像:
\newcommand{\ProfilePhoto}{assets/portrait.jpg}

% ====== 渲染宏:在 main.tex 中调用 \BasicSection ======
% 使用 tabularx 确保名称、联系方式、头像并排显示
\newcommand{\BasicSection}{%
	\begin{tabularx}{\textwidth}{X p{0.18\textwidth}}
		{\LARGE\bfseries \ProfileName\par}
				{\normalsize 电话:\ProfilePhone \quad 邮箱:\href{mailto:\ProfileEmail}{\ProfileEmail}\par}
				{\normalsize 网站:\href{\ProfileWebsite}{\ProfileWebsite} \quad GitHub:\href{\ProfileGitHub}{\ProfileGitHub}\par}
				{\normalsize \textbf{求职意向:} \ProfileTarget \quad \textbf{申请方向:} \ProfileDirection\par}
				&
				\raggedleft\vspace*{-2.4\baselineskip}\IfFileExists{\ProfilePhoto}{%
					\includegraphics[width=\linewidth,keepaspectratio,height=2.8cm]{\ProfilePhoto}%
				}{}%
	\end{tabularx}
	\vspace{1pt}
}

% 额外说明:可选信息默认注释,不会显示;若希望开启某一项,取消对应行的注释并填写内容即可。
