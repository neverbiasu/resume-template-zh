% research.tex - 科研或项目经历(编辑此文件)
% 用途:列出科研、项目或课题,包含研究内容、方法与成果等要点。
\renewcommand{\ResearchSection}{%
  \section*{科研/项目经历}
  \entry{可控人脸风格化}{负责人}{线上和线下}{2024.6 -- 2024.8}{%
    \item 核心内容:在 Diffusion 模型中加入风格提取、人脸 ID 编码,并通过交叉注意力注入到 UNet,结合图像打标器与 ControlNet 实现高质量人脸风格迁移
    \item 技术亮点:风格提取器、ID 编码器、VLM/LLM 文本特征与 ControlNet 选择策略,结合 SDXL 管道与 IP 适配器进行特征注入
  \item 成果:Project Page 已部署于 GitHub Pages;已被 VINCI 接收(科研/竞赛收录)
  }

  \entry{启智序章——自动片头生成}{负责人}{校内(本科)}{2024.2 -- 2024.8}{%
    \item 核心内容:开发自动生成课程介绍视频的软件,整合 LLMs、TTS、I2I 及 A2F 技术,使用 LangChain、ChatGLM3、GPTSoVITS、ComfyUI、SadTalker 等工具
    \item 成果:项目获计算机设计大赛省一等奖并在后续推进国创立项
  }
  % 已发表论文(示例已发布链接)
  \entry{国际会议论文}{Springer}{会议章节}{2024}{%
    \item 论文链接:\url{https://link.springer.com/chapter/10.1007/978-981-99-7587-7_8}
  }
}
